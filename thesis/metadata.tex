

%%% Choose a language %%%

\newif\ifEN
\ENtrue   % uncomment this for english
%\ENfalse   % uncomment this for czech

%%% Configuration of the title page %%%

\def\ThesisTitleStyle{mff} % MFF style
%\def\ThesisTitleStyle{cuni} % uncomment for old-style with cuni.cz logo
%\def\ThesisTitleStyle{natur} % uncomment for nature faculty logo

\def\UKFaculty{Faculty of Mathematics and Physics}
%\def\UKFaculty{Faculty of Science}

\def\UKName{Charles University in Prague} % this is not used in the "mff" style

% Thesis type names, as used in several places in the title
%\def\ThesisTypeTitle{\ifEN BACHELOR THESIS \else BAKALÁŘSKÁ PRÁCE \fi}
\def\ThesisTypeTitle{\ifEN MASTER THESIS \else DIPLOMOVÁ PRÁCE \fi}
%\def\ThesisTypeTitle{\ifEN RIGOROUS THESIS \else RIGORÓZNÍ PRÁCE \fi}
%\def\ThesisTypeTitle{\ifEN DOCTORAL THESIS \else DISERTAČNÍ PRÁCE \fi}
%\def\ThesisGenitive{\ifEN bachelor \else bakalářské \fi}
\def\ThesisGenitive{\ifEN master \else diplomové \fi}
%\def\ThesisGenitive{\ifEN rigorous \else rigorózní \fi}
%\def\ThesisGenitive{\ifEN doctoral \else disertační \fi}
%\def\ThesisAccusative{\ifEN bachelor \else bakalářskou \fi}
\def\ThesisAccusative{\ifEN master \else diplomovou \fi}
%\def\ThesisAccusative{\ifEN rigorous \else rigorózní \fi}
%\def\ThesisAccusative{\ifEN doctoral \else disertační \fi}



%%% Fill in your details %%%

% (Note: \xxx is a "ToDo label" which makes the unfilled visible. Remove it.)
\def\ThesisTitle{Efficient hyperparameter optimization}
\def\ThesisAuthor{Bc.~Jiří Krejčí}
\def\YearSubmitted{2024}

% department assigned to the thesis
\def\Department{Department of Theoretical Computer Science and Mathematical Logic}
% Is it a department (katedra), or an institute (ústav)?
\def\DeptType{Department}

\def\Supervisor{doc.~Mgr.~Martin Pilát,~Ph.D.}
\def\SupervisorsDepartment{Department of Theoretical Computer Science and Mathematical Logic}

% Study programme and specialization
\def\StudyProgramme{Computer Science}
\def\StudyBranch{Artificial Intelligence (IUIP)}

\def\Dedication{%
I am grateful for the support and patience of my family, my partner, and my friends on this arduous journey. I also appreciate all the support and great advice of my supervisor Martin Pilát. Thank you.

A special mention goes to the e-INFRA CZ project (ID:90254), supported by the Ministry of Education, Youth and Sports of the Czech Republic, for providing the computational resources.
}

\def\AbstractEN{%
Hyperparameter optimization significantly impacts model performance and substantial effort went into the development of robust and efficient algorithms for this task. Our research found that several new algorithms utilizing partial evaluations have been published recently. However, it is not clear from the literature how the algorithms perform in various scenarios. In this thesis, we compared the leading algorithms through experiments on diverse tasks, including tabular benchmarks and real-world deep-learning problems, with a special focus on healthcare datasets. The results show that the recent multi-fidelity techniques outperform random search. Nevertheless, no single algorithm consistently excelled across all problems, highlighting the need for ongoing comparison studies in hyperparameter optimization.
}

\def\AbstractCS{%
Ladění hyperparametrů má značný vliv na vlastnosti výsledného modelu a proto bylo věnováno vývoji robustních a efektivních algoritmů pro tuto úlohu velké úsilí. Nedávno bylo vyvinuto několik nových algoritmů využívajících částečných vyhodnocení optimalizované funkce. Z literatury nicméně není zřejmé, jak si tyto algoritmy vedou na rozmanitých problémech. V této diplomové práci jsme experimentálně srovnali aktuální algoritmy pro ladění hyperparametrů v mnoha úlohách. Tyto úlohy se skládaly z tabulárních benchmarků a reálných úloh hlubokého učení, včetně datových sad z oblasti zdravotnictví. Výsledky ukazují, že nedávné multi-fidelity techniky dosahují lepších výsledků než náhodné prohledávání. Přesto však žádný algoritmus nepodával konzistentně nejlepší výkon ve všech problémech, což zdůrazňuje potřebu průběžných srovnávacích studií v oblasti optimalizace hyperparametrů.
}

% 3 to 5 keywords (recommended), each enclosed in curly braces.
% Keywords are useful for indexing and searching for the theses by topic.
\def\Keywords{%
{deep learning} {hyperparameter optimization} {Bayesian optimization} {multi-fidelity}
}

% If your abstracts are long and do not fit in the infopage, you can make the
% fonts a bit smaller by this setting. (Also, you should try to compress your abstract more.)
% Alternatively, consider increasing the size of the page by uncommenting the
% geometry modification in thesis.tex.
\def\InfoPageFont{}
%\def\InfoPageFont{\small}  %uncomment to decrease font size

